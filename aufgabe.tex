Ein interstellares Bauunternehmen baut eine riesige Raumstation in Form eines Mondes. Eine
Lagerhaltung ist nicht nötig, das Material wird just-in-time geliefert.
Folgende Daten sollen erfasst werden:
Das große Projekt der Raumstation soll in Teilprojekte (z.B. Abluftschacht) aufgeteilt werden.
Diese haben jeweils einen eigenen Materialbedarf, eine Arbeitsstundenschätzung, Budget und
zugewiesene Mitarbeiter. Mitarbeiter mit Wochenarbeitsstunden, Anstellungsart (Vollzeit,
Teilzeit, Werkstudent), Name, Adresse, Position (Wachposten, Arbeiter, Supervisor), etc. Lieferanten für Baumaterial, Lieferungen mit Material, Mengen, Preisen und Projekt.
Falls Unfälle, mit möglicherweise katastrophalem Ausgang, passieren, sollen diese mit zugehörigem Teilprojekt, beteiligten Mitarbeitern und die Folgen wie z.B. Verletzungen oder Sachschäden erfasst werden. Auch der zuständige Supervisor muss angegeben sein.
Mitarbeiter tragen Ihre tatsächlich erbrachte Arbeitszeit pro Teilprojekt jeden Tag in eine
Tabelle ein. Hierüber wird der Projektfortschritt ermittelt.
Der Bauherr interessiert sich besonders für folgende Informationen:
• Wann wird die Raumstation voraussichtlich fertiggestellt sein? (Errechenbar mit bisheriger Arbeitszeit / Woche und Restzeit der Projekte)
• Welche Teilprojekte haben ihr Budget überschritten?
• Welches Baumaterial wird besonders oft benötigt?
• Wie viele Teilprojekte sind abgeschlossen?